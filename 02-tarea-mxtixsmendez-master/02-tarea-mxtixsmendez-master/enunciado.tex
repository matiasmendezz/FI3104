\documentclass[letter, 11pt]{article}
%% ================================
%% Packages =======================
\usepackage[utf8]{inputenc}      %%
\usepackage[T1]{fontenc}         %%
\usepackage{lmodern}             %%
\usepackage[spanish]{babel}      %%
\decimalpoint                    %%
\usepackage{fullpage}            %%
\usepackage{fancyhdr}            %%
\usepackage{graphicx}            %%
\usepackage{amsmath}             %%
\usepackage{color}               %%
\usepackage{mdframed}            %%
\usepackage[colorlinks]{hyperref}%%
%% ================================
%% ================================

%% ================================
%% Page size/borders config =======
\setlength{\oddsidemargin}{0in}  %%
\setlength{\evensidemargin}{0in} %%
\setlength{\marginparwidth}{0in} %%
\setlength{\marginparsep}{0in}   %%
\setlength{\voffset}{-0.5in}     %%
\setlength{\hoffset}{0in}        %%
\setlength{\topmargin}{0in}      %%
\setlength{\headheight}{54pt}    %%
\setlength{\headsep}{1em}        %%
\setlength{\textheight}{8.5in}   %%
\setlength{\footskip}{0.5in}     %%
%% ================================
%% ================================

%% =============================================================
%% Headers setup, environments, colors, etc.
%%
%% Header ------------------------------------------------------
\fancypagestyle{firstpage}
{
  \fancyhf{}
  \lhead{\includegraphics[height=4.5em]{LogoDFI.jpg}}
  \rhead{FI3104-1 \semestre\\
         Métodos Numéricos para la Ciencia e Ingeniería\\
         Prof.: \profesor}
  \fancyfoot[C]{\thepage}
}

\pagestyle{plain}
\fancyhf{}
\fancyfoot[C]{\thepage}
%% -------------------------------------------------------------
%% Environments -------------------------------------------------
\newmdenv[
  linecolor=gray,
  fontcolor=gray,
  linewidth=0.2em,
  topline=false,
  bottomline=false,
  rightline=false,
  skipabove=\topsep
  skipbelow=\topsep,
]{ayuda}
%% -------------------------------------------------------------
%% Colors ------------------------------------------------------
\definecolor{gray}{rgb}{0.5, 0.5, 0.5}
%% -------------------------------------------------------------
%% Aliases ------------------------------------------------------
\newcommand{\scipy}{\texttt{scipy}}
%% -------------------------------------------------------------
%% =============================================================
%% =============================================================================
%% CONFIGURACION DEL DOCUMENTO =================================================
%% Llenar con la información pertinente al curso y la tarea
%%
\newcommand{\tareanro}{2}
\newcommand{\fechaentrega}{17/10/2020 21:59 hrs}
\newcommand{\semestre}{2020B}
\newcommand{\profesor}{Valentino González}
%% =============================================================================
%% =============================================================================


\begin{document}
\thispagestyle{firstpage}

\begin{center}
  {\uppercase{\LARGE \bf Tarea \tareanro}}\\
  Fecha de entrega: \fechaentrega
\end{center}


%% =============================================================================
%% ENUNCIADO ===================================================================
\noindent{\large \bf Problema 1}

Considere un péndulo simple en un medio viscoso. La ecuación de movimiento es
la siguiente:

$$ \ddot{\phi} = -\gamma \dot{\phi} - \frac{g}{l} \sin(\phi)$$

\noindent donde $\gamma$ es un coeficiente de fricción, $g$ es la aceleración
de gravedad y $l$ es el largo del péndulo. En el caso de pequeñas oscilaciones
se puede aproximar $\sin(\phi)\sim\phi$ y existe una solución analítica para el
problema.

Considere $l=5.XXX\ {\rm m}$ y $\gamma=2.XXX\ {\rm s^{-1}}$ donde $XXX$ son los 3
últimos dígitos de su RUT.

\begin{itemize}

  \item Integre la ecuación de movimiento considerando condiciones de borde del
    estilo $\phi(t=0) = \phi_0$ y $\dot{\phi}(t=0) = 0$. Primero considere un
    caso $\phi_0$ pequeño y compruebe que la solución es parecida al caso de
    pequeñas oscilaciones (solución analítica).

  \item Ahora integre nuevamente para $\phi_0=\pi/(2.XXX)$. Compare con la
    solución analítica para el problema de pequeñas oscilaciones. En
    particular, estudie la energía del péndulo. ¿Cuál solución pierde energía
    más rápido? ¿la solución real o la solución de pequeñas oscilaciones?

\end{itemize}

En esta pregunta Ud. debe implementar el método de $RK4$. Discuta cómo eligió
el paso temporal y qué precisión espera alcanzar.


\vspace{1.5em}
\noindent{\large \bf Problema 2}

El sistema de Lorenz es un set de ecuaciones diferenciales ordinarias conocido
por tener algunas soluciones caóticas, la más famosa, el llamado atractor de
Lorenz. El sistema de ecuaciones es el siguiente:

\begin{flalign*}
  \dfrac{dx}{ds} &= \sigma (y - x)\\
  \dfrac{dy}{ds} &= x (\rho - z) - y\\
  \dfrac{dz}{ds} &= xy - \beta z
\end{flalign*}

La solución más famosa se obtiene con los parámetros $\sigma=10$, $\beta=8/3$ y
$\rho=28$. Utilice esos parámetros, elija un set de condiciones iniciales
$(x_0, y_0, z_0)$ e integre la ecuación por un tiempo que estime conveniente.
Esta vez se pide que utilice un algoritmo $RK4$ pero no necesita implementarlo,
puede usar los algoritmos disponibles en \texttt{scipy.integrate} o cualquier
otro que encuentre y que sea de uso libre.  Grafique la solución en 3D: $(x(t),
y(t), z(t))$.

A continuación perturbe la condición inicial $y_0 \rightarrow (y_0 + 0.0XXX)$,
donde $XXX$ son los tres últimos dígitos de su RUT. Comente sobre el
comportamiento de la solución en comparación con el obtenido con las primeras
condiciones iniciales escogidas.


\begin{ayuda}
  \small
  \noindent {\bf Ayuda.}
  Este repositorio incluye el archivo \texttt{3D.py} que demuestra como hacer
  un plot en 3D usando \texttt{matplotlib}. Puede usar este archivo como guía
  de ayuda.
\end{ayuda}


%% FIN ENUNCIADO ===============================================================
%% =============================================================================
\vspace{1em}
\noindent{\bf Instrucciones importantes.}
\begin{itemize}

  \item Utilice \texttt{git} durante el desarrollo de la tarea para mantener un
    historial de los cambios realizados. La siguiente
    \href{https://education.github.com/git-cheat-sheet-education.pdf}{cheat
      sheet} le puede ser útil. \textbf{Revisaremos el uso apropiado
    de la herramienta y asignaremos una fracción del puntaje a este ítem.}
    Realice cambios pequeños y guarde su progreso (a través de \emph{commits})
    regularmente. No guarde código que no corre o compila (si lo hace por algún
    motivo deje un mensaje claro que lo indique). Escriba mensajes claros que
    permitan hacerse una idea de lo que se agregó y/o cambió de un
    \texttt{commit} al siguiente.

  \item Revisaremos su uso correcto de \texttt{python}. Si define una función
    relativametne larga o con muchos parámetros, recuerde escribir el
    \emph{docstring} que describa los parámetros que recibe la función, el
    output, y el detalle de qué es lo que hace la función. Recuerde que
    generalmente es mejor usar varias funciones cortas (que hagan una sola cosa
    bien) que una muy larga (que lo haga todo).  Utilice nombres explicativos
    tanto para las funciones como para las variables de su código.  El mejor
    nombre es aquel que permite entender qué hace la función sin tener que leer
    su implementación.

  \item También evaluaremos que su código apruebe la guía de estilo sintáctico
    \href{https://www.python.org/dev/peps/pep-0008/}{\texttt{PEP8}}. En
    \href{http://pep8online.com}{esta página} puede chequear si su código
    aprueba \texttt{PEP8}.

  \item La tarea se entrega subiendo su trabajo a github. Clone este
    repositorio (el que está en su propia cuenta privada), trabaje en el código
    y en el informe y cuando haya terminado asegúrese de hacer un último
    \texttt{commit} y luego un \texttt{push} para subir todo su trabajo a
    github.

  \item El informe debe ser entregado en formato \texttt{pdf}, este debe ser
    claro sin información de más ni de menos. \textbf{Esto es muy importante,
    no escriba de más, esto no mejorará su nota sino que al contrario}. La
    presente tarea probablemente no requiere informes de más de 5 páginas en
    total (esto sólo una referencia útil).  Asegúrese de utilizar figuras
    efectivas y/o tablas para resumir sus resultados. \textbf{Revise su
    ortografía}.

  \item No olvide indicar su RUT en el informe.

  \item Repartición de puntaje: 40\% implementación y resolución del problema
    (independiente de la calidad de su código); 45\% calidad del reporte
    entregado: demuestra comprensión del problema y su solución, claridad del
    lenguaje, calidad de las figuras utilizadas; 5\% aprueba a no
    \texttt{PEP8}; 10\% diseño del código: modularidad, uso efectivo de nombres
    de variables y funciones, docstrings, \underline{uso de git}, etc.

\end{itemize}
\end{document}

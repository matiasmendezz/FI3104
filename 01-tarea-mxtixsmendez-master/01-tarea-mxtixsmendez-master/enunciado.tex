\documentclass[letter, 11pt]{article}
%% ================================
%% Packages =======================
\usepackage[utf8]{inputenc}      %%
\usepackage[T1]{fontenc}         %%
\usepackage{lmodern}             %%
\usepackage[spanish]{babel}      %%
\decimalpoint                    %%
\usepackage{fullpage}            %%
\usepackage{fancyhdr}            %%
\usepackage{graphicx}            %%
\usepackage{amsmath}             %%
\usepackage{color}               %%
\usepackage{mdframed}            %%
\usepackage[colorlinks]{hyperref}%%
%% ================================
%% ================================

%% ================================
%% Page size/borders config =======
\setlength{\oddsidemargin}{0in}  %%
\setlength{\evensidemargin}{0in} %%
\setlength{\marginparwidth}{0in} %%
\setlength{\marginparsep}{0in}   %%
\setlength{\voffset}{-0.5in}     %%
\setlength{\hoffset}{0in}        %%
\setlength{\topmargin}{0in}      %%
\setlength{\headheight}{54pt}    %%
\setlength{\headsep}{1em}        %%
\setlength{\textheight}{8.5in}   %%
\setlength{\footskip}{0.5in}     %%
%% ================================
%% ================================

%% =============================================================
%% Headers setup, environments, colors, etc.
%%
%% Header ------------------------------------------------------
\fancypagestyle{firstpage}
{
  \fancyhf{}
  \lhead{\includegraphics[height=4.5em]{LogoDFI.jpg}}
  \rhead{FI3104-1 \semestre\\
         Métodos Numéricos para la Ciencia e Ingeniería\\
         Prof.: \profesor}
  \fancyfoot[C]{\thepage}
}

\pagestyle{plain}
\fancyhf{}
\fancyfoot[C]{\thepage}
%% -------------------------------------------------------------
%% Environments -------------------------------------------------
\newmdenv[
  linecolor=gray,
  fontcolor=gray,
  linewidth=0.2em,
  topline=false,
  bottomline=false,
  rightline=false,
  skipabove=\topsep
  skipbelow=\topsep,
]{ayuda}
%% -------------------------------------------------------------
%% Colors ------------------------------------------------------
\definecolor{gray}{rgb}{0.5, 0.5, 0.5}
%% -------------------------------------------------------------
%% Aliases ------------------------------------------------------
\newcommand{\scipy}{\texttt{scipy}}
%% -------------------------------------------------------------
%% =============================================================
%% =============================================================================
%% CONFIGURACION DEL DOCUMENTO =================================================
%% Llenar con la información pertinente al curso y la tarea
%%
\newcommand{\tareanro}{1}
\newcommand{\fechaentrega}{26/09/2020 21:59 hrs}
\newcommand{\semestre}{2020B}
\newcommand{\profesor}{Valentino González}
%% =============================================================================
%% =============================================================================


\begin{document}
\thispagestyle{firstpage}

\begin{center}
  {\uppercase{\LARGE \bf Tarea \tareanro}}\\
  Fecha de entrega: \fechaentrega
\end{center}


%% =============================================================================
%% ENUNCIADO ===================================================================
\noindent{\large \bf Problema}

En esta tarea crearemos una función que calcule la función {\it percentil},
un problema común en estadística (muy usado para interpretar resultados
experimentales).

Consideremos un experimento cuyo resultado es un número real que podemos tratar
como si fuese aleatorio dadas las incertidumbres asociadas al experimento. A la
probabilidad asociada a cada posible resultado $x$ ---en realidad un intervalo
infinitesimal alrededor de un resultado $x$--- se le llama la función densidad
de probabilidad $pdf(x)$. De modo que $pdf(x) dx$ es la probabilidad de que al
realizar el experimento se obtenga como resultado el un valor entre $x$ y
$x+dx$.

Notemos que, entonces, la probabilidad de obtener un valor menor que $a$ se
puede calcular como:

$$p(x<a) = \int_{-\infty}^{a} pdf(x) dx $$

Una pregunta común que surge al realizar un experimento es qué valor de $a$ me
asegura que esa probabilidad sea mayor que, por ejemplo, 95\%. A eso se le
llama la función percentil evaluada en 0.95 y corresponde a resolver
\underline{para $a$} el siguiente problema:

$$ 0.95 = \int_{-\infty}^{a} pdf(x) dx $$

En esta tarea consideraremos para la $pdf$ una distribución $\chi^2$:

$$ pdf(x) = \chi^2(x) = \frac{1}{2^{k/2}\Gamma(k/2)} x^{k/2-1} e^{-x/2} $$

donde $\Gamma$ es la función Gamma:

$$ \Gamma(z) = \int_0^{\infty} x^{z-1}e^{-x}dx $$

$\chi^2$ tiene un parámetro: $k$. Para la tarea Ud. debe considerar $k=4.XXX$,
donde $XXX$ son los 3 últimos dígitos de su RUT antes del guión.

{\bf Calcule el valor de $a$ que resuelve $p(x<a) = 0.95$.}

{\bf Ud. debe escribir su propia función que calcule $\chi^2(x)$, las integrales, y
el algoritmo que resuelve la ecuación. No use librerías.}

\begin{ayuda}
  \small
  \noindent {\bf Sugerencias.}

  Esta es una serie de \underline{sugerencias} sobre cómo resolver el problema.
  No es estrictamente necesario seguirlas.

  \begin{enumerate}
    \item Escriba una función que calcule $\Gamma(x)$. Para comprobar que su
      función hace lo correcto, compruebe que $\Gamma(n) = (n-1)!$. ¿Cómo
      resolvió el problema del $\infty$ en el límite de integración?

    \item Escriba otra función que calcule $\chi^2(x)$.

    \item Escriba una función que calcule $f(a) = \int_0^{a} \chi^2(x) dx $.
      Preocúpese de que la integral sea resuelta con buena precisión. Este es
      un asunto importante a discutir en el informe. ¿Qué precisión eligió?

    \item Escriba un algoritmo que resuelva $f(a) = 0.95$. ¿Qué algoritmo
      utilizó y por qué? Si el algoritmo requiere algún parámetro: ¿qué
      parámetros eligió y cómo los eligió?

    \item En el informe se recomienda discutir temas como la precisión del
      cálculo, el tiempo de ejecución, e incluir gráficos o tablas que
      demuestren el resultado.
  \end{enumerate}

\end{ayuda}
%% FIN ENUNCIADO ===============================================================
%% =============================================================================

\vspace{1em}
\noindent{\bf Otras instrucciones importantes.}
\begin{itemize}

  \item Utilice \texttt{git} durante el desarrollo de la tarea para mantener un
    historial de los cambios realizados. La siguiente
    \href{https://education.github.com/git-cheat-sheet-education.pdf}{cheat
    sheet} le puede ser útil. Esto no será evaluado esta vez pero evaluaremos
    el use efectivo de \texttt{git} en el futuro, así que empiece a usarlo.

  \item La tarea se entrega como un \texttt{push} simple a su repositorio
    privado. El \texttt{push} debe incluir todos los códigos usados además de
    su informe.

  \item El informe debe ser entregado en formato \texttt{pdf}, este debe ser
    claro sin información de más ni de menos. Esto es importante, no escriba de
    más, esto no mejorará su nota sino que al contrario. 4 o 5 páginas son un
    largo razonable para la presente tarea.  Asegúrese de utilizar figuras
    efectivas y/o tablas para resumir sus resultados. Revise su ortografía.

  \item No olvide indicar su RUT en el informe.

\end{itemize}



\end{document}

\documentclass[letter, 11pt]{article}
%% ================================
%% Packages =======================
\usepackage[utf8]{inputenc}      %%
\usepackage[T1]{fontenc}         %%
\usepackage{lmodern}             %%
\usepackage[spanish]{babel}      %%
\decimalpoint                    %%
\usepackage{fullpage}            %%
\usepackage{fancyhdr}            %%
\usepackage{graphicx}            %%
\usepackage{amsmath}             %%
\usepackage{color}               %%
\usepackage{mdframed}            %%
\usepackage[colorlinks]{hyperref}%%
%% ================================
%% ================================

%% ================================
%% Page size/borders config =======
\setlength{\oddsidemargin}{0in}  %%
\setlength{\evensidemargin}{0in} %%
\setlength{\marginparwidth}{0in} %%
\setlength{\marginparsep}{0in}   %%
\setlength{\voffset}{-0.5in}     %%
\setlength{\hoffset}{0in}        %%
\setlength{\topmargin}{0in}      %%
\setlength{\headheight}{54pt}    %%
\setlength{\headsep}{1em}        %%
\setlength{\textheight}{8.5in}   %%
\setlength{\footskip}{0.5in}     %%
%% ================================
%% ================================

%% =============================================================
%% Headers setup, environments, colors, etc.
%%
%% Header ------------------------------------------------------
\fancypagestyle{firstpage}
{
  \fancyhf{}
  \lhead{\includegraphics[height=4.5em]{LogoDFI.jpg}}
  \rhead{FI3104-1 \semestre\\
         Métodos Numéricos para la Ciencia e Ingeniería\\
         Prof.: \profesor}
  \fancyfoot[C]{\thepage}
}

\pagestyle{plain}
\fancyhf{}
\fancyfoot[C]{\thepage}
%% -------------------------------------------------------------
%% Environments -------------------------------------------------
\newmdenv[
  linecolor=gray,
  fontcolor=gray,
  linewidth=0.2em,
  topline=false,
  bottomline=false,
  rightline=false,
  skipabove=\topsep
  skipbelow=\topsep,
]{ayuda}
%% -------------------------------------------------------------
%% Colors ------------------------------------------------------
\definecolor{gray}{rgb}{0.5, 0.5, 0.5}
%% -------------------------------------------------------------
%% Aliases ------------------------------------------------------
\newcommand{\scipy}{\texttt{scipy}}
%% -------------------------------------------------------------
%% =============================================================
%% =============================================================================
%% CONFIGURACION DEL DOCUMENTO =================================================
%% Llenar con la información pertinente al curso y la tarea
%%
\newcommand{\tareanro}{05}
\newcommand{\fechaentrega}{09/12/2020 21:59 hrs}
\newcommand{\semestre}{2020B}
\newcommand{\profesor}{Valentino González}
%% =============================================================================
%% =============================================================================


\begin{document}
\thispagestyle{firstpage}

\begin{center}
  {\uppercase{\LARGE \bf Tarea \tareanro}}\\
  Fecha de entrega: \fechaentrega
\end{center}


%% =============================================================================
%% ENUNCIADO ===================================================================
\noindent{\large \bf Problema 1} (Código e Informe)

Estime las posición del centro de masa de un sólido descrito por la
intersección de un toro y un cilindro dados por las siguientes ecuaciones:

\begin{align}
  {\rm Toro} : z^2 + \left( \sqrt{x^2 + y^2} - 3 \right)^2 &\leq 1\\
  {\rm Cilindro} : (x - 2)^2 + z^2 &\leq 1
\end{align}

La densidad del sólido varía según la siguiente ecuación:

$$ \rho(x, y, z) = 0.5 \times (x^2 + y^2 + z^2) $$

Resuelva el problema usando alguno de los métodos de integración de Monte
Carlo, para ello debe definirse un volúmen, ojalá lo más pequeño posible, que
englobe al sólido sobre el cual quiere integrar.

Debido a la naturaleza aleatoria del algoritmo, la determinación que haga tiene
una incertidumbre asociada. Para estimar la incertidumbre, repita el
experimento un número grande de veces y reporte el valor promedio y su
desviación estándar. Discuta en su informe cómo se comporta el error en función
del número de puntos utilizados para la estimación del volumen.

\begin{ayuda}
  \small
  \noindent {\bf Ayuda.}

  Le puede ser útil mirar la sección ``Simple Monte Carlo Integration'' del
  libro ``Numerical Reicpes in C''.

\end{ayuda}

\vspace{1em}
\noindent{\large \bf Problema 2} (Código y gráfico solamente)

Se desea obtener una muestra aleatoria de números con distribución densidad de
probabilidad proporcional a la siguiente función:

$$
W(x) = \frac{1}{5\sqrt{\pi}} \left( 2 e^{-(x-1.5)^2} + 3 e^{-(x+1)^2} \right)
$$

Utilice el algoritmo de Metrópolis partiendo de un $x_0$ escogido a gusto entre
-2 y 2 y con una distribución propuesta $x_p = x_n + \delta \times r$, donde r
es una variable aleatoria de la distribución uniforme U(-1, 1). La variable
$\delta$ tiene un valor fijo que Ud. debe determinar. De acuerdo a la
literatura, un buen valor  de $\delta$ es aquel para el cual se aceptan
aproximadamente 50\% de las proposiciones.

Genere una muestra de unos 10 millones de puntos. Para comprobar que su
resultado es adecuado grafique $W(x)$ y un histograma de sus variables
aleatorias ambos apropiadamente normalizados.

Determine la incertidumbre asociada a cada caja (bin) del histograma y \textbf{
grafique el histograma con las barras de error asociadas}.

\begin{ayuda}
  \small
  \noindent {\bf Ayuda.}

  La estrategia sugerida para determinar los errores asociados es realizar una
  simulación estilo Monte Carlo: repita muchas veces (N = 100 o más) el
  procedimiento completo del problema, pero cada vez utilice una semilla
  distinta, o un punto de partida distintos o ambos. El resultado es que
  obtendrá N histogramas que deben ser distintos debido a la naturaleza
  aleatoria del procedimiento. Para un bin dado, con N valores distintos, una
  forma de definir el tamaño de la barra de error es hacerla igual a la
  desviación estándar de de los N valores para ese bin.

\end{ayuda}


%%%%%%%%%%%%%%%%%%%%%%%%%%%%%%%%%%%%%%%%%%%%%%%%%%%%%%%%%%%%%%%%%%%%%%%%%%%%%%%
%%%%%%%%%%%%%%%%%%%%%%%%%%%%%%%%%%%%%%%%%%%%%%%%%%%%%%%%%%%%%%%%%%%%%%%%%%%%%%%

\vspace{2em}
\noindent\textbf{Instrucciones Importantes.}
\begin{itemize}

\item Para reducir la cantidad de trabajo necesaria para completar esta tarea,
  sólo debe escribir un informe para el Problema 1. De todos modos debe
  entregar el código escrito para ambos problemas y el gráfico solicitado en el
  Problema 2

 \item Repartición de puntaje:

   \subitem - 30\% Código Problema 1. Implementación y resolución del problema.
     
    \subitem - 30\% Informe Problema 1. Demuestra comprensión del problema y su
    solución, claridad del lenguaje, calidad de las figuras y/o tablas
    utilizadas. Para esta tarea el informe probablemente no require más de 3
    páginas pero esto es sólo una referencia.

    \subitem - 30\% Código y gráfico Problema 2. Implementación y resolución
    del problema.

    \subitem - 5\% Códigos aprueban a no \texttt{PEP8}.

    \subitem - 5\% Diseño del código: modularidad, uso efectivo de nombres de
    variables y funciones, docstrings, \underline{uso de git}, etc


\item Evaluaremos su uso correcto de \texttt{python}. Si define una función
  relativametne larga o con muchos parámetros, recuerde escribir el
  \emph{docstring} que describa los parámetros que recibe la función, el
  output, y el detalle de qué es lo que hace la función. Recuerde que
  generalmente es mejor usar varias funciones cortas (que hagan una sola cosa
  bien) que una muy larga (que lo haga todo).  Utilice nombres explicativos
  tanto para las funciones como para las variables de su código. El mejor
  nombre es aquel que permite entender qué hace la función sin tener que leer
  su implementación ni su \emph{docstring}.

\item Su código debe aprobar la guía sintáctica de estilo
  (\href{https://www.python.org/dev/peps/pep-0008/}{\texttt{PEP8}}). En
  \href{http://pep8online.com}{esta página} puede chequear si su código aprueba
  \texttt{PEP8}.

\item Utilice \texttt{git} durante el desarrollo de la tarea para mantener un
  historial de los cambios realizados. La siguiente
  \href{https://education.github.com/git-cheat-sheet-education.pdf}{cheat
    sheet} le puede ser útil. {\bf Revisaremos el uso apropiado de la
  herramienta y asignaremos una fracción del puntaje a este ítem.} Realice
  cambios pequeños y guarde su progreso (a través de \emph{commits})
  regularmente. No guarde código que no corre o compila (si lo hace por algún
  motivo deje un mensaje claro que lo indique). Escriba mensajes claros que
  permitan hacerse una idea de lo que se agregó y/o cambió de un
  \texttt{commit} al siguiente.

\item Al hacer el informe usted debe decidir qué es interesante y agregar las
  figuras correspondientes. No olvide anotar los ejes, las unidades e incluir
  una \emph{caption} o título que describa el contenido de cada figura.

\item La tarea se entrega subiendo su trabajo a github. Trabaje en el código y
  en el informe, haga \textit{commits} regulares y cuando haya terminado
  asegúrese de hacer un último \texttt{commit} y luego un \texttt{push} para
  subir todo su trabajo a github. \textbf{REVISE SU REPOSITORIO PARA ASEGURARSE
  QUE SUBIÓ LA TAREA. SI UD. NO PUEDE VER SU INFORME EN GITHUB.COM, TAMPOCO
PODREMOS NOSOTROS.}

\item El informe debe ser entregado en formato \texttt{pdf}, este debe ser
  claro sin información de más ni de menos. \textbf{Esto es muy importante, no
  escriba de más, esto no mejorará su nota sino que al contrario}. La presente
  tarea probablemente no requiere informes de más de 3 páginas. Asegúrese de
  utilizar figuras efectivas y tablas para resumir sus resultados.

\item \textbf{REVISE SU ORTOGRAFÍA.}


\end{itemize}

%% FIN ENUNCIADO ===============================================================
%% =============================================================================

\end{document}
